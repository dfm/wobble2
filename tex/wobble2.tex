% The layout of this document has been shamelessly stolen from the MIT licensed project:
% https://github.com/davidwhogg/GaussianProductRefactor
% by David W. Hogg, Adrian Price-Whelan, and Boris Leistedt

\documentclass[10pt]{article}

\usepackage{amsmath, bm, mathrsfs, amssymb}
\usepackage[dvipsnames]{xcolor}
\usepackage[hidelinks,
            colorlinks=true,
            linkcolor=NavyBlue,
            citecolor=darkgray,
            urlcolor=NavyBlue]{hyperref}
\usepackage{pifont}
\usepackage{graphicx}
\usepackage{doi}

\usepackage{natbib}
\bibliographystyle{dfm_abbrvnat}
\setcitestyle{round,citesep={,},aysep={}}

% text stuhh
\newcommand{\documentname}{\textsl{Note}}
\newcommand{\sectionname}{Section}
\newcommand{\equationname}{equation}
\newcommand{\notename}{Note}
\renewcommand{\tablename}{Table}
\newcommand{\acronym}[1]{{\small{#1}}}
\newcommand{\code}[1]{\texttt{#1}}
\newcommand{\foreign}[1]{\textsl{#1}}
\newcommand{\etal}{\foreign{et~al.}}

% math stuhh
\newcommand{\hquad}{~~}
\newcommand{\given}{\,|\,}
\newcommand{\dd}{\mathrm{d}}
\newcommand{\T}{^{\!\mathsf{T}\!}}
\newcommand{\inv}{^{-1}}
\newcommand{\scalar}[1]{#1}
\renewcommand{\vector}[1]{\boldsymbol{#1}}
\newcommand{\tensor}[1]{\mathbf{#1}}
\renewcommand{\matrix}[1]{\mathsf{#1}}
\newcommand{\normal}{\mathcal{N}\!\,}

% variables
\newcommand{\celerite}{\code{celerite}}

% Journal keys - kill me now
\newcommand{\aj}{Astron.\,J.}
\newcommand{\apj}{Astrophys.\,J.} % Astrophysical Journal
\newcommand{\apjs}{Astrophys.\,J.\,Supp.\,Ser.} % Astrophysical Journal Supplements

% page layout stuff
\setlength{\topmargin}{0.0in}
\setlength{\headheight}{0.0in}
\setlength{\headsep}{0.0in}
\setlength{\parindent}{\baselineskip}
\setlength{\textwidth}{4.3in}
\setlength{\textheight}{2\textwidth}
\raggedbottom\sloppy\sloppypar\frenchspacing

% this might be crazy, but here's my number
\setlength{\marginparsep}{0.15in}
\setlength{\marginparwidth}{2.7in}
\usepackage{marginfix} % necessary but possibly evil
\newcounter{marginnote}
\setcounter{marginnote}{0}
\renewcommand{\footnote}[1]{\refstepcounter{marginnote}\textsuperscript{\themarginnote}\marginpar{\color{darkgray}\raggedright\footnotesize\textsuperscript{\themarginnote}#1}}
\newcommand{\tfigurerule}{\rule{0pt}{1ex}\\ \rule{\marginparwidth}{0.5pt}\\ \rule{0pt}{0.25ex}}
\newcommand{\bfigurerule}{\rule{0pt}{0.25ex}\\ \rule{\marginparwidth}{0.5pt}\\ \rule{0pt}{1ex}}
\renewcommand{\caption}[1]{\parbox{\marginparwidth}{\footnotesize\refstepcounter{figure}\textbf{\figurename~\thefigure}: {#1}}}

% and make the left margin correct
\setlength{\oddsidemargin}{0.5\paperwidth}
\addtolength{\oddsidemargin}{-1.0in}
\addtolength{\oddsidemargin}{-0.5\textwidth}
\addtolength{\oddsidemargin}{-0.5\marginparwidth}
\addtolength{\oddsidemargin}{-0.5\marginparsep}

\begin{document}

\section*{An insane version of wobble + psoap + etc.}

\noindent\textbf{Daniel Foreman-Mackey}\\
{\footnotesize%
\textsl{Center for Computational Astrophysics, Flatiron Institute, New York, NY, USA}%
}

\medskip\noindent\textbf{Megan Bedell}\\
{\footnotesize%
\textsl{Center for Computational Astrophysics, Flatiron Institute, New York, NY, USA}%
}

\medskip\noindent\textbf{et al.}\\
{\footnotesize%
\textsl{Affiliations}\\
\textsl{More affiliations}%
}

\paragraph{Abstract:}

It's wild.

\section{The basics}

\paragraph{PSOAP}

From \citet{Czekala:2017}.
The idea here was to model all the fluxes in the spectral time series as a sum of GP models where each kernel represented a different component of the system (stars, tellurics, etc.).
The model is evaluated by shifting the wavelength of each component into its rest frame and then evaluating the kernel there.
This is awesome: it's directly modeling the data (no interpolation) assuming only that the spectrum should be smooth.
But, it's also slow (can't use celerite or similar) and (probably) restricted to additive components.

\paragraph{wobble}

From \citet{Bedell:2019}.
Here, the model parameters are the flux values for each spectral component on a fine grid that is interpolated into the observed frame and optimized.
Much more flexible (can easily incorporate any number of additive and multiplicative components) and demonstrated that these high-dimensional optimizations could work.
But, the results are sensitive to grid sizes and regularization parameters, and it's not trivial to incorporate time variability in the spectral shape.

\paragraph{The new thing}

The idea here is to sort of combine these ideas into something that is both flexible and (hopefully!) tractable.
The model parameters are now (like in wobble) the flux as a function of wavelength for each component, but instead of evaluated on a grid, these will be evaluated \emph{at the locations of the observations}.
Therefore, we will have $N$ parameters for each data point (where $N$ is the number of spectral components), and the effective wavelength of each flux parameter will be a function of the RV offsets for each component at each time.
This is mildly insane, but probably possible.
Then, we will use a GP prior on the parameters to enforce a PSOAP-like smoothness constraint on each spectral component.
Then, the model is evaluated as a sum and/or product of each spectral component, and optimized using some likelihood function (possibly including things like baseline normalization, etc.) in the observed space.

This has the benefits of PSOAP: we don't need to choose a grid or regularization parameters.
But it also has the flexibility of wobble (+ some!): we can support any number of spectral components with different observational properties.
Since our flux parameters each have both a time and a rest frame wavelength associated with them, our GP prior can be a function of time and wavelength: enforcing smooth variability in both dimensions.
And, I think I know how to build this model using celerite kernels, so we can probably be fast too!

\paragraph{Open questions}

Is a GP model actually a good model for the spectral dimension?
Is there some way we could make it better?

% Render the references
\clearpage\raggedright
\bibliography{refs}

\end{document}
